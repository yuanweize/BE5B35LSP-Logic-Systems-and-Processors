% ==============================================================================
% Chapter 3: RS latch + Shannon expansion - "traffic light" rule
% ==============================================================================
\section{RS Latch (Traffic-Light Rule)}

\begin{hack}[NOR-gate RS - active-high]
\textbf{Traffic-light memory trick:}
\begin{itemize}
\item $S=1$ $\to$ $Q$ becomes 1 (Set)
\item $R=1$ $\to$ $Q$ becomes 0 (Reset)
\item Both 0 $\to$ \textbf{copy the previous state}
\item Both 1 $\to$ write ``forbidden'' / ``unstable''
\end{itemize}
\end{hack}

\begin{hack}[NAND-gate RS - active-low]
\textbf{It is the opposite!}
\begin{itemize}
\item $\bar{S}=0$ $\to$ $Q$ becomes 1
\item $\bar{R}=0$ $\to$ $Q$ becomes 0
\item Both 1 $\to$ \textbf{copy the previous state}
\item Both 0 $\to$ forbidden
\end{itemize}
\end{hack}

\subsection{RS truth table (must memorize)}

\begin{center}
\begin{tabular}{|cc|c|l|}
\hline
\textbf{S} & \textbf{R} & \textbf{Q} & \textbf{What to write} \\
\hline
0 & 0 & Q & copy previous \\
0 & 1 & 0 & write 0 \\
1 & 0 & 1 & write 1 \\
1 & 1 & ? & write ``forbidden'' \\
\hline
\end{tabular}
\end{center}

\subsection{How to solve waveform problems}

\begin{hack}[Three steps for timing diagrams]
\begin{enumerate}
\item Draw vertical lines at every change point on S and R
\item For each time segment, read the values of S and R
\item Fill in Q using the truth table
\end{enumerate}

\textbf{Mnemonic:} S high $\Rightarrow$ Q high, R high $\Rightarrow$ Q low, both low $\Rightarrow$ copy!
\end{hack}

% ==============================================================================
\section{Shannon Expansion (Copy-and-Paste Method)}
% ==============================================================================

\begin{hack}[Shannon expansion - just apply the formula]
\textbf{Do not derive it. Use the template:}
\[
F = \bar{A} \cdot F_0 + A \cdot F_1
\]

\textbf{Three steps:}
\begin{enumerate}
\item $F_0$: replace every $A$ with 0
\item $F_1$: replace every $A$ with 1
\item Plug into the formula
\end{enumerate}
\end{hack}

\subsection{Shannon expansion example}

\begin{example}[Example: $F = AB + \bar{A}C + BC$, expand w.r.t. A]
\textbf{Step 1: compute $F_0$ (A=0)}

Replace $A \to 0$ and $\bar{A} \to 1$:
\begin{align*}
F_0 &= (0)B + (1)C + BC \\
	&= 0 + C + BC \\
	&= C \quad \text{(absorption law)}
\end{align*}

\textbf{Step 2: compute $F_1$ (A=1)}

Replace $A \to 1$ and $\bar{A} \to 0$:
\begin{align*}
F_1 &= (1)B + (0)C + BC \\
	&= B + 0 + BC \\
	&= B \quad \text{(absorption law)}
\end{align*}

\textbf{Step 3: plug into the formula}
\[
\boxed{F = \bar{A} \cdot C + A \cdot B}
\]
\end{example}

\begin{hack}[Absorption law quick notes]
\begin{itemize}
\item $X + XY = X$ (keep the bigger term)
\item $X + \bar{X}Y = X + Y$ (complement trick)
\end{itemize}
\end{hack}

\subsection{Two-variable Shannon expansion}

\begin{hack}[Expand w.r.t. AB together]
\textbf{Formula:}
\[
F = \bar{A}\bar{B}F_{00} + \bar{A}BF_{01} + A\bar{B}F_{10} + ABF_{11}
\]

\textbf{Procedure:} substitute (A,B)=(0,0), (0,1), (1,0), (1,1).
\end{hack}

\subsection{Shannon expansion as a MUX}

\begin{keybox}[Shannon = 2:1 MUX]
$F = \bar{A} \cdot F_0 + A \cdot F_1$ corresponds to:
\begin{itemize}
\item select = $A$
\item input 0 = $F_0$
\item input 1 = $F_1$
\end{itemize}
\end{keybox}

\begin{pitfall}[Common Shannon mistakes]
\begin{itemize}
\item When substituting $A=0$, remember $\bar{A}=1$ (don't swap them)
\item Don't forget absorption when simplifying
\item In a MUX, connect $I_0$ to $F_0$ and $I_1$ to $F_1$ (order matters)
\end{itemize}
\end{pitfall}
