% ==============================================================================
% Chapter 1: Number Systems - Calculator-first, avoid binary
% ==============================================================================
\section{Number Systems (Calculator-first)}

\begin{hack}[Core idea: compute in decimal, convert at the end]
\textbf{Don't waste time doing the whole thing in binary.}

Compute in decimal first, and handle overflow only at the last step.
\end{hack}

\subsection{Powers of Two (memorize)}

\begin{center}
\begin{tabular}{|c|c||c|c||c|c|}
\hline
$2^4$ & 16 & $2^8$ & 256 & $2^{12}$ & 4096 \\
\hline
$2^5$ & 32 & $2^9$ & 512 & $2^{14}$ & 16384 \\
\hline
$2^6$ & 64 & $2^{10}$ & 1024 & $2^{16}$ & 65536 \\
\hline
$2^7$ & 128 & $2^{11}$ & 2048 & $2^{20}$ & 1M \\
\hline
\end{tabular}
\end{center}

\subsection{N-bit ranges (must know)}

\begin{formula}[Range quick notes]
\begin{itemize}
\item \textbf{Unsigned:} $[0, 2^N-1]$
\item \textbf{Signed:} $[-2^{N-1}, 2^{N-1}-1]$
\end{itemize}

\textbf{Common:}
\begin{itemize}
\item 8-bit signed: $[-128, +127]$
\item 10-bit signed: $[-512, +511]$
\item 16-bit signed: $[-32768, +32767]$
\end{itemize}
\end{formula}

\subsection{Unsigned overflow}

\begin{hack}[Unsigned overflow - take modulo]
\textbf{Rule:} result = (decimal result) \% $2^N$

\vspace{3pt}
\textbf{Example:} 8-bit unsigned, compute $200 + 100$
\begin{enumerate}
\item Decimal: $200 + 100 = 300$
\item Modulo: $300 \mod 256 = 300 - 256 = 44$
\item \textbf{Answer: 44}
\end{enumerate}

That's it. No need to convert to binary.
\end{hack}

\subsection{Signed overflow}

\begin{hack}[Signed overflow - subtract $2^N$ if above max]
\textbf{Three-step recipe:}
\begin{enumerate}
\item Compute the decimal result $R$
\item Check: $R > 2^{N-1}-1$ (max positive)?
\item If yes: $R - 2^N$ is the final answer
\end{enumerate}

\vspace{3pt}
\textbf{Example:} 10-bit signed, compute $511 + 511$
\begin{enumerate}
\item Decimal: $511 + 511 = 1022$
\item Check: $1022 > 511$? Yes, overflow.
\item Fix: $1022 - 1024 = -2$
\item \textbf{Answer: $-2$}
\end{enumerate}

\textbf{Symmetric case:} if $R < -2^{N-1}$, then add $2^N$.
\end{hack}

\subsection{Negative numbers to two's complement}

\begin{hack}[Two's complement for negatives - no bit-flip + 1 needed]
\textbf{Pro formula:} two's complement of $-X$ is $2^N - X$

\vspace{3pt}
\textbf{Example:} 8-bit representation of $-5$
\begin{enumerate}
\item Compute: $256 - 5 = 251$
\item Convert to binary: $251 = 128+64+32+16+8+2+1$
\item \textbf{Answer: \texttt{11111011}}
\end{enumerate}

Much faster than ``invert bits + 1''.
\end{hack}

\subsection{Two's complement to decimal}

\begin{hack}[Use MSB to determine sign]
\textbf{Check the most significant bit:}
\begin{itemize}
\item MSB=0: positive, convert normally
\item MSB=1: negative, use the formula
\end{itemize}

\textbf{Negative formula:} value = (unsigned binary value) $- 2^N$

\vspace{3pt}
\textbf{Example:} 8-bit \texttt{11110100}
\begin{enumerate}
\item MSB=1, so it's negative
\item Treat as unsigned: $128+64+32+16+4 = 244$
\item Subtract $2^8$: $244 - 256 = -12$
\item \textbf{Answer: $-12$}
\end{enumerate}
\end{hack}

\subsection{Sign extension}

\begin{hack}[Extending bit-width - copy the sign bit]
\textbf{Unsigned:} pad with 0s

\textbf{Signed:} replicate MSB
\begin{itemize}
\item Positive (MSB=0): pad with 0s
\item Negative (MSB=1): pad with 1s
\end{itemize}

\textbf{Example:} 4-bit $\to$ 8-bit
\begin{itemize}
\item \texttt{0110} (+6) $\to$ \texttt{0000 0110}
\item \texttt{1010} (-6) $\to$ \texttt{1111 1010}
\end{itemize}
\end{hack}

\begin{pitfall}[Don't lose free points]
\begin{itemize}
\item Read carefully: signed vs unsigned
\item Watch the bit-width (8-bit vs 10-bit ranges differ)
\item Use the right powers of two: $2^{10}=1024$, not 1000
\end{itemize}
\end{pitfall}
