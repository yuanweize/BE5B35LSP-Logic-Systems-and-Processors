% ==============================================================================
% Kapitola 3: RS záchyt + Shannonův rozklad - pravidlo semaforu
% ==============================================================================
\section{RS záchyt (pravidlo semaforu)}

\begin{hack}[RS s NOR - aktivní v 1]
\textbf{Pomůcka se semaforem:}
\begin{itemize}
\item $S=1$ $\to$ $Q$ je 1 (Set)
\item $R=1$ $\to$ $Q$ je 0 (Reset)
\item Obě 0 $\to$ \textbf{opiš předchozí stav}
\item Obě 1 $\to$ napiš ``zakázáno'' / ``nestabilní''
\end{itemize}
\end{hack}

\begin{hack}[RS s NAND - aktivní v 0]
\textbf{Je to obráceně!}
\begin{itemize}
\item $\bar{S}=0$ $\to$ $Q$ je 1
\item $\bar{R}=0$ $\to$ $Q$ je 0
\item Obě 1 $\to$ \textbf{opiš předchozí stav}
\item Obě 0 $\to$ zakázáno
\end{itemize}
\end{hack}

\subsection{Tabulka RS (nutno umět)}

\begin{center}
\begin{tabular}{|cc|c|l|}
\hline
\textbf{S} & \textbf{R} & \textbf{Q} & \textbf{Co napsat} \\
\hline
0 & 0 & Q & opiš předchozí \\
0 & 1 & 0 & napiš 0 \\
1 & 0 & 1 & napiš 1 \\
1 & 1 & ? & napiš ``zakázáno'' \\
\hline
\end{tabular}
\end{center}

\subsection{Postup na časové průběhy}

\begin{hack}[Tři kroky]
\begin{enumerate}
\item Na průbězích S a R \textbf{nakresli svislé čáry} v každém bodě změny
\item Pro každý interval přečti hodnoty S a R
\item Doplň Q podle tabulky
\end{enumerate}

\textbf{Pomůcka:} S vysoko $\Rightarrow$ Q vysoko, R vysoko $\Rightarrow$ Q nízko, obě nízko $\Rightarrow$ opiš!
\end{hack}

% ==============================================================================
\section{Shannonův rozklad (metoda copy-paste)}
% ==============================================================================

\begin{hack}[Shannon - jen dosaď do vzorce]
\textbf{Neodvozuj. Použij šablonu:}
\[
F = \bar{A} \cdot F_0 + A \cdot F_1
\]

\textbf{Tři kroky:}
\begin{enumerate}
\item $F_0$: ve výrazu dej všude $A=0$
\item $F_1$: ve výrazu dej všude $A=1$
\item Dosadíš do vzorce
\end{enumerate}
\end{hack}

\subsection{Praktický příklad}

\begin{example}[Příklad: $F = AB + \bar{A}C + BC$, rozklad podle A]
\textbf{Krok 1: $F_0$ (A=0)}

Dosadíš $A \to 0$ a $\bar{A} \to 1$:
\begin{align*}
F_0 &= (0)B + (1)C + BC \\
	&= 0 + C + BC \\
	&= C \quad \text{(absorpce)}
\end{align*}

\textbf{Krok 2: $F_1$ (A=1)}

Dosadíš $A \to 1$ a $\bar{A} \to 0$:
\begin{align*}
F_1 &= (1)B + (0)C + BC \\
	&= B + 0 + BC \\
	&= B \quad \text{(absorpce)}
\end{align*}

\textbf{Krok 3: dosazení do vzorce}
\[
\boxed{F = \bar{A} \cdot C + A \cdot B}
\]
\end{example}

\begin{hack}[Rychlá absorpce]
\begin{itemize}
\item $X + XY = X$ (větší člen stačí)
\item $X + \bar{X}Y = X + Y$ (doplňkové pravidlo)
\end{itemize}
\end{hack}

\subsection{Shannon pro dvě proměnné}

\begin{hack}[Rozklad podle AB současně]
\textbf{Vzorec:}
\[
F = \bar{A}\bar{B}F_{00} + \bar{A}BF_{01} + A\bar{B}F_{10} + ABF_{11}
\]

\textbf{Postup:} dosaď (A,B)=(0,0), (0,1), (1,0), (1,1).
\end{hack}

\subsection{Shannon jako MUX}

\begin{keybox}[Shannon = 2:1 multiplexer]
$F = \bar{A} \cdot F_0 + A \cdot F_1$ odpovídá:
\begin{itemize}
\item výběr = $A$
\item vstup 0 = $F_0$
\item vstup 1 = $F_1$
\end{itemize}
\end{keybox}

\begin{pitfall}[Časté chyby]
\begin{itemize}
\item Když dáš $A=0$, tak $\bar{A}=1$ (nepoplést)
\item Při zjednodušování nezapomeň na absorpci
\item U MUXu platí $I_0=F_0$ a $I_1=F_1$ (pořadí je důležité)
\end{itemize}
\end{pitfall}
