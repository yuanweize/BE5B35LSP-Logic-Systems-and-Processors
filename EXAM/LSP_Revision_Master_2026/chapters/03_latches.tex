% ==============================================================================
% 第三章: RS锁存器 + 香农展开 - 交通灯法则
% ==============================================================================
\section{RS锁存器 (交通灯法则)}

\begin{hack}[NOR门RS - 高电平有效]
\textbf{红绿灯记忆法:}
\begin{itemize}
\item $S=1$ $\to$ $Q$变1 (Set=设置)
\item $R=1$ $\to$ $Q$变0 (Reset=清零)
\item 都是0 $\to$ \textbf{抄前一个状态}
\item 都是1 $\to$ 写``禁止''或``不稳定''
\end{itemize}
\end{hack}

\begin{hack}[NAND门RS - 低电平有效]
\textbf{反过来!}
\begin{itemize}
\item $\bar{S}=0$ $\to$ $Q$变1
\item $\bar{R}=0$ $\to$ $Q$变0
\item 都是1 $\to$ \textbf{抄前一个状态}
\item 都是0 $\to$ 禁止
\end{itemize}
\end{hack}

\subsection{RS真值表 (必背)}

\begin{center}
\begin{tabular}{|cc|c|l|}
\hline
\textbf{S} & \textbf{R} & \textbf{Q} & \textbf{怎么写} \\
\hline
0 & 0 & Q & 抄上一个 \\
0 & 1 & 0 & 写0 \\
1 & 0 & 1 & 写1 \\
1 & 1 & ? & 写``禁止'' \\
\hline
\end{tabular}
\end{center}

\subsection{波形分析做题法}

\begin{hack}[波形题三步走]
\begin{enumerate}
\item 在S和R波形上\textbf{画竖线}标出每个变化点
\item 每段时间看S和R的值
\item 按真值表填Q的值
\end{enumerate}

\textbf{口诀:}S高Q高,R高Q低,都低抄前面!
\end{hack}

% ==============================================================================
\section{香农展开 (复制粘贴法)}
% ==============================================================================

\begin{hack}[香农展开 - 复制粘贴!]
\textbf{不要推导!直接套公式:}
\[
F = \bar{A} \cdot F_0 + A \cdot F_1
\]

\textbf{三步操作:}
\begin{enumerate}
\item $F_0$:把式子里的$A$全部\textbf{擦掉换成0}
\item $F_1$:把式子里的$A$全部\textbf{擦掉换成1}
\item 套公式写答案
\end{enumerate}
\end{hack}

\subsection{香农展开实战}

\begin{example}[例:$F = AB + \bar{A}C + BC$,对A展开]
\textbf{Step 1: 算$F_0$ (A=0)}

把$A$换成0,$\bar{A}$换成1:
\begin{align*}
F_0 &= (0)B + (1)C + BC \\
    &= 0 + C + BC \\
    &= C \quad \text{(吸收律)}
\end{align*}

\textbf{Step 2: 算$F_1$ (A=1)}

把$A$换成1,$\bar{A}$换成0:
\begin{align*}
F_1 &= (1)B + (0)C + BC \\
    &= B + 0 + BC \\
    &= B \quad \text{(吸收律)}
\end{align*}

\textbf{Step 3: 套公式}
\[
\boxed{F = \bar{A} \cdot C + A \cdot B}
\]
\end{example}

\begin{hack}[吸收律速记]
\begin{itemize}
\item $X + XY = X$ (有大的就不要小的)
\item $X + \bar{X}Y = X + Y$ (互补相加)
\end{itemize}
\end{hack}

\subsection{两变量香农展开}

\begin{hack}[对AB同时展开]
\textbf{公式:}
\[
F = \bar{A}\bar{B}F_{00} + \bar{A}BF_{01} + A\bar{B}F_{10} + ABF_{11}
\]

\textbf{操作:}分别代入(A,B)=(0,0), (0,1), (1,0), (1,1)
\end{hack}

\subsection{香农展开做MUX}

\begin{keybox}[香农 = 2:1 MUX]
$F = \bar{A} \cdot F_0 + A \cdot F_1$ 直接对应:
\begin{itemize}
\item 选择信号 = $A$
\item 输入0 = $F_0$
\item 输入1 = $F_1$
\end{itemize}
\end{keybox}

\begin{pitfall}[香农展开易错点]
\begin{itemize}
\item $\bar{A}$代入0时变成1,别搞反!
\item 化简时别忘了吸收律
\item MUX的$I_0$接$F_0$,$I_1$接$F_1$,顺序别错
\end{itemize}
\end{pitfall}
