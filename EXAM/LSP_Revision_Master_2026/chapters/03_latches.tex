% ==============================================================================
% 第三章: RS锁存器与香农展开 - 详细推导
% ==============================================================================
\section{RS锁存器与香农展开}

\subsection{RS锁存器基础}

\begin{formula}[RS锁存器真值表]
\begin{center}
\begin{tabular}{|cc|c|l|}
\hline
\textbf{S} & \textbf{R} & \textbf{Q(next)} & \textbf{状态} \\
\hline
0 & 0 & Q & 保持 \\
0 & 1 & 0 & 复位 \\
1 & 0 & 1 & 置位 \\
1 & 1 & ? & 禁止/不稳定 \\
\hline
\end{tabular}
\end{center}

\textbf{特征方程:} $Q_{next} = S + \bar{R}Q$ (约束: $SR = 0$)
\end{formula}

\begin{pitfall}[S=R=1 的问题]
当 $S=R=1$ 时:
\begin{itemize}
\item NOR型锁存器: $Q=\bar{Q}=0$ (矛盾!)
\item NAND型锁存器: $Q=\bar{Q}=1$ (矛盾!)
\item 释放后状态不确定 (竞争条件)
\end{itemize}
\textbf{考试中:} 通常标记为"禁止"或"不稳定"
\end{pitfall}

\subsection{NOR型 vs NAND型}

\begin{algorithm}[两种RS锁存器对比]
\textbf{NOR型RS锁存器:}
\begin{itemize}
\item $Q = \overline{R + \bar{Q}}$
\item $\bar{Q} = \overline{S + Q}$
\item 高电平有效 (S=1置位, R=1复位)
\item 禁止态: S=R=1
\end{itemize}

\textbf{NAND型RS锁存器:}
\begin{itemize}
\item $Q = \overline{\bar{S} \cdot \bar{Q}}$
\item 低电平有效 ($\bar{S}$=0置位, $\bar{R}$=0复位)
\item 禁止态: $\bar{S}$=$\bar{R}$=0
\end{itemize}
\end{algorithm}

\subsection{RS锁存器状态分析}

\begin{example}[典型例题: 状态序列分析]
\textbf{给定:} NOR型RS锁存器,初始 $Q=0$

\textbf{输入序列:} $(S,R) = (1,0) \to (0,0) \to (0,1) \to (0,0) \to (1,1)$

\textbf{Step-by-Step分析:}

\begin{center}
\small
\begin{tabular}{|c|cc|c|l|}
\hline
\textbf{时刻} & \textbf{S} & \textbf{R} & \textbf{Q} & \textbf{说明} \\
\hline
$t_0$ & - & - & 0 & 初始状态 \\
$t_1$ & 1 & 0 & 1 & 置位: $Q \to 1$ \\
$t_2$ & 0 & 0 & 1 & 保持: $Q$ 不变 \\
$t_3$ & 0 & 1 & 0 & 复位: $Q \to 0$ \\
$t_4$ & 0 & 0 & 0 & 保持: $Q$ 不变 \\
$t_5$ & 1 & 1 & ? & 禁止态 \\
\hline
\end{tabular}
\end{center}
\end{example}

% ==============================================================================
\subsection{香农展开 - 核心考点}
% ==============================================================================

\begin{formula}[香农展开定理]
任意布尔函数 $F(x_1, x_2, ..., x_n)$ 可以对变量 $x_i$ 展开:
\[
F = x_i \cdot F_{x_i=1} + \bar{x_i} \cdot F_{x_i=0}
\]

其中:
\begin{itemize}
\item $F_{x_i=1}$ = 将 $x_i$ 代入1后的子函数 (正因子)
\item $F_{x_i=0}$ = 将 $x_i$ 代入0后的子函数 (负因子)
\end{itemize}
\end{formula}

\begin{algorithm}[香农展开步骤 (手把手)]
\textbf{目标:} 对 $F(A,B,C)$ 关于变量 $A$ 进行香农展开

\textbf{步骤:}
\begin{enumerate}
\item 计算正因子: 令 $A=1$,得到 $F_1 = F|_{A=1}$
\item 计算负因子: 令 $A=0$,得到 $F_0 = F|_{A=0}$
\item 组合结果: $F = A \cdot F_1 + \bar{A} \cdot F_0$
\end{enumerate}
\end{algorithm}

\begin{example}[香农展开完整例题]
\textbf{题目:} 对 $F = AB + \bar{A}C + BC$ 关于 $A$ 进行香农展开

\textbf{Step 1: 计算正因子 $F_1$ (令 $A=1$)}
\begin{align*}
F_1 &= F|_{A=1} \\
&= (1)B + \overline{(1)}C + BC \\
&= B + 0 \cdot C + BC \\
&= B + BC \\
&= B \quad \text{(吸收律)}
\end{align*}

\textbf{Step 2: 计算负因子 $F_0$ (令 $A=0$)}
\begin{align*}
F_0 &= F|_{A=0} \\
&= (0)B + \overline{(0)}C + BC \\
&= 0 + 1 \cdot C + BC \\
&= C + BC \\
&= C \quad \text{(吸收律)}
\end{align*}

\textbf{Step 3: 组合香农展开式}
\[
\boxed{F = A \cdot B + \bar{A} \cdot C}
\]

\textbf{验证:} 展开结果 $AB + \bar{A}C$

原式 $AB + \bar{A}C + BC$ 可化简:
\begin{align*}
&= AB + \bar{A}C + BC(A + \bar{A}) \\
&= AB + \bar{A}C + ABC + \bar{A}BC \\
&= AB(1+C) + \bar{A}C(1+B) \\
&= AB + \bar{A}C \quad \checkmark
\end{align*}
\end{example}

\subsection{多变量香农展开}

\begin{algorithm}[多变量展开]
对两个变量 $A, B$ 同时展开:
\[
F = AB \cdot F_{11} + A\bar{B} \cdot F_{10} + \bar{A}B \cdot F_{01} + \bar{A}\bar{B} \cdot F_{00}
\]

其中 $F_{ij} = F|_{A=i, B=j}$
\end{algorithm}

\begin{example}[两变量香农展开]
\textbf{题目:} 对 $F = ABC + \bar{A}\bar{B}C + A\bar{C}$ 关于 $A,B$ 展开

\textbf{计算四个因子:}
\begin{itemize}
\item $F_{11} = F|_{A=1,B=1} = (1)(1)C + 0 + (1)\bar{C} = C + \bar{C} = 1$
\item $F_{10} = F|_{A=1,B=0} = 0 + 0 + \bar{C} = \bar{C}$
\item $F_{01} = F|_{A=0,B=1} = 0 + 0 + 0 = 0$
\item $F_{00} = F|_{A=0,B=0} = 0 + (1)(1)C + 0 = C$
\end{itemize}

\textbf{结果:}
\[
F = AB \cdot 1 + A\bar{B} \cdot \bar{C} + \bar{A}B \cdot 0 + \bar{A}\bar{B} \cdot C
\]
\[
\boxed{F = AB + A\bar{B}\bar{C} + \bar{A}\bar{B}C}
\]
\end{example}

\subsection{香农展开与MUX实现}

\begin{keybox}[用MUX实现布尔函数]
香农展开直接对应 2:1 MUX:
\begin{itemize}
\item 选择信号 = 展开变量
\item $I_0$ 输入 = 负因子 $F_0$
\item $I_1$ 输入 = 正因子 $F_1$
\end{itemize}

\textbf{级联:} 对每个因子继续展开 $\to$ 多级MUX树
\end{keybox}

\begin{pitfall}[香农展开常见错误]
\begin{itemize}
\item 代入时忘记处理 $\bar{A}$ (应变成1或0)
\item 化简时漏掉吸收律
\item MUX输入接反 ($F_0$ 和 $F_1$ 位置)
\item 多变量展开时因子数量算错 ($2^n$ 个)
\end{itemize}
\end{pitfall}
