% ==============================================================================
% 第四章: 流水线 - 连连看
% ==============================================================================
\section{流水线 (连连看)}

\begin{hack}[流水线关键词匹配]
\textbf{看到题目关键词,直接连答案:}

\begin{center}
\begin{tabular}{|l|l|}
\hline
\textbf{题目说} & \textbf{答案写} \\
\hline
Data Hazard & Forwarding \\
\hline
Load-Use & Stall (气泡) \\
\hline
Branch/Jump & Flush (冲刷) \\
\hline
Structural & 加硬件 \\
\hline
\end{tabular}
\end{center}
\end{hack}

\subsection{5级流水线 (背下来)}

\begin{formula}[五个阶段]
\begin{center}
\texttt{IF $\to$ ID $\to$ EX $\to$ MEM $\to$ WB}
\end{center}

\begin{tabular}{|c|l|l|}
\hline
IF & Instruction Fetch & 取指令 \\
ID & Instruction Decode & 译码/读寄存器 \\
EX & Execute & ALU计算 \\
MEM & Memory & 访存 \\
WB & Write Back & 写回 \\
\hline
\end{tabular}
\end{formula}

\subsection{流水线时序图模板}

\begin{hack}[画图照着抄!]
\begin{verbatim}
     C1  C2  C3  C4  C5  C6  C7
I1   IF  ID  EX  MEM WB
I2       IF  ID  EX  MEM WB
I3           IF  ID  EX  MEM WB
I4               IF  ID  EX  MEM WB
\end{verbatim}

\textbf{规律:}每条指令往右错一格
\end{hack}

\subsection{三种Hazard}

\begin{keybox}[Hazard分类]
\textbf{1. 结构冲突 (Structural):}
\begin{itemize}
\item 硬件不够用 (如只有一个内存端口)
\item 解决:加硬件
\end{itemize}

\textbf{2. 数据冲突 (Data):}
\begin{itemize}
\item 后面指令要用前面还没算完的数据
\item 解决:Forwarding 或 Stall
\end{itemize}

\textbf{3. 控制冲突 (Control):}
\begin{itemize}
\item 分支跳转导致取错指令
\item 解决:Flush/预测
\end{itemize}
\end{keybox}

\subsection{RAW冲突}

\begin{hack}[RAW冲突检测]
\textbf{看寄存器!}
\begin{verbatim}
I1: ADD R1, R2, R3   ; 写R1
I2: SUB R4, R1, R5   ; 读R1 <- RAW!
\end{verbatim}

\textbf{判断方法:}后面指令读的寄存器 = 前面指令写的寄存器?

是 $\to$ RAW冲突!
\end{hack}

\subsection{Forwarding路径}

\begin{hack}[Forwarding两条路]
\textbf{1. EX/MEM $\to$ EX:}
\begin{itemize}
\item 前一条ALU结果直接给后一条用
\item 解决1周期RAW
\end{itemize}

\textbf{2. MEM/WB $\to$ EX:}
\begin{itemize}
\item 前两条的结果给当前用
\item 解决2周期RAW
\end{itemize}
\end{hack}

\subsection{Load-Use必须Stall}

\begin{pitfall}[Forwarding救不了Load-Use!]
\begin{verbatim}
LW  R1, 0(R2)   ; MEM阶段才有数据
ADD R3, R1, R4  ; EX阶段就要数据
\end{verbatim}

\textbf{必须插入1个Stall(气泡)!}
\begin{verbatim}
     C1  C2  C3  C4  C5  C6
LW   IF  ID  EX  MEM WB
ADD      IF  ID  --  EX  MEM
\end{verbatim}

``--''就是气泡/Stall
\end{pitfall}

\subsection{CPI计算}

\begin{hack}[CPI公式]
\[
\text{CPI} = 1 + \text{Stall率}
\]

\textbf{例:}30\%指令是Load,其中50\%造成Stall
\begin{align*}
\text{Stall率} &= 0.3 \times 0.5 = 0.15 \\
\text{CPI} &= 1 + 0.15 = 1.15
\end{align*}
\end{hack}

\subsection{加速比}

\begin{formula}[加速比公式]
\[
\text{Speedup} = \frac{nk}{k + n - 1} \to k
\]

$k$ = 流水线级数,$n$ = 指令数

当$n$很大时,加速比接近$k$倍
\end{formula}

\begin{hack}[流水线题目套路]
\begin{enumerate}
\item 画时序图
\item 找冲突(看寄存器)
\item 标Forwarding箭头或Stall
\item 算CPI
\end{enumerate}
\end{hack}
