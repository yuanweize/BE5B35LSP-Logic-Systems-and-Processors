% ==============================================================================
% 第四章: 处理器流水线 - 重点考点
% ==============================================================================
\section{处理器流水线}

\subsection{流水线基础}

\begin{formula}[流水线阶段 (5级经典)]
\begin{center}
\begin{tabular}{|c|l|l|}
\hline
\textbf{阶段} & \textbf{全称} & \textbf{功能} \\
\hline
IF & Instruction Fetch & 取指令 \\
ID & Instruction Decode & 译码/读寄存器 \\
EX & Execute & 执行/ALU计算 \\
MEM & Memory Access & 访存 \\
WB & Write Back & 写回寄存器 \\
\hline
\end{tabular}
\end{center}
\end{formula}

\subsection{流水线时序图}

\begin{example}[流水线执行时序]
\textbf{指令序列:} I1, I2, I3, I4, I5

\begin{center}
\small
\texttt{%
\begin{tabular}{l|ccccccccc}
     & C1 & C2 & C3 & C4 & C5 & C6 & C7 & C8 & C9 \\
\hline
I1 & IF & ID & EX & MEM & WB &    &    &    &    \\
I2 &    & IF & ID & EX & MEM & WB &    &    &    \\
I3 &    &    & IF & ID & EX & MEM & WB &    &    \\
I4 &    &    &    & IF & ID & EX & MEM & WB &    \\
I5 &    &    &    &    & IF & ID & EX & MEM & WB \\
\end{tabular}
}
\end{center}

\textbf{性能:} 5条指令用9个周期 (理想情况)
\end{example}

\subsection{流水线Hazard分类}

\begin{keybox}[三类Hazard]
\textbf{1. 结构冲突 (Structural Hazard):}
\begin{itemize}
\item 硬件资源不足 (如单端口内存)
\item 解决: 增加硬件资源
\end{itemize}

\textbf{2. 数据冲突 (Data Hazard):}
\begin{itemize}
\item RAW (Read After Write) - 最常见
\item WAW (Write After Write)
\item WAR (Write After Read)
\end{itemize}

\textbf{3. 控制冲突 (Control Hazard):}
\begin{itemize}
\item 分支指令导致
\item 解决: 预测/延迟槽/Stall
\end{itemize}
\end{keybox}

\subsection{数据冲突详解}

\begin{algorithm}[RAW冲突检测]
\textbf{条件:} 指令 $I_j$ 读取指令 $I_i$ 要写的寄存器,且 $I_i$ 还没写回

\textbf{例:}
\begin{verbatim}
I1: ADD R1, R2, R3   ; 写 R1
I2: SUB R4, R1, R5   ; 读 R1 <- RAW!
\end{verbatim}

\textbf{时序分析:}
\begin{itemize}
\item I1 在 C5(WB) 才写回 R1
\item I2 在 C3(ID) 就要读 R1
\item 冲突: 读发生在写之前!
\end{itemize}
\end{algorithm}

\subsection{Forwarding (旁路)}

\begin{formula}[Forwarding路径]
\textbf{EX/MEM $\to$ EX:}
\begin{itemize}
\item 将ALU结果直接送到下一条指令的ALU输入
\item 解决1周期RAW冲突
\end{itemize}

\textbf{MEM/WB $\to$ EX:}
\begin{itemize}
\item 将访存结果或旧ALU结果送到ALU输入
\item 解决2周期RAW冲突
\end{itemize}
\end{formula}

\begin{example}[Forwarding实例]
\textbf{指令:}
\begin{verbatim}
I1: ADD R1, R2, R3
I2: SUB R4, R1, R5
I3: AND R6, R1, R7
\end{verbatim}

\textbf{时序 (有Forwarding):}
\begin{center}
\small
\texttt{%
\begin{tabular}{l|ccccc}
     & C1 & C2 & C3 & C4 & C5 \\
\hline
I1 & IF & ID & EX & MEM & WB \\
I2 &    & IF & ID & EX* & MEM \\
I3 &    &    & IF & ID & EX** \\
\end{tabular}
}
\end{center}

\begin{itemize}
\item *EX: 从I1的EX/MEM寄存器forward
\item **EX: 从I1的MEM/WB寄存器forward
\end{itemize}

\textbf{无需Stall!}
\end{example}

\subsection{Load-Use Hazard}

\begin{pitfall}[Load后立即使用]
\textbf{问题:} LW指令在MEM阶段才得到数据

\begin{verbatim}
I1: LW  R1, 0(R2)   ; MEM阶段才有R1
I2: ADD R3, R1, R4  ; EX阶段需要R1
\end{verbatim}

\textbf{即使有Forwarding也必须Stall 1周期!}

\begin{center}
\small
\texttt{%
\begin{tabular}{l|cccccc}
     & C1 & C2 & C3 & C4 & C5 & C6 \\
\hline
LW  & IF & ID & EX & MEM & WB &    \\
ADD &    & IF & ID & -- & EX & MEM \\
\end{tabular}
}
\end{center}

"--" 表示插入的 Stall (气泡)
\end{pitfall}

\subsection{Stall与性能计算}

\begin{formula}[CPI计算]
\[
\text{CPI} = \text{CPI}_{ideal} + \text{Stall cycles per instruction}
\]

\textbf{理想CPI} = 1 (每周期完成一条指令)

\textbf{实际CPI} = 1 + (数据Stall + 控制Stall)
\end{formula}

\begin{example}[性能计算例题]
\textbf{给定:}
\begin{itemize}
\item 30\% 的指令是Load
\item 50\% 的Load后面紧跟使用其结果的指令
\item 每次Load-Use造成1周期Stall
\end{itemize}

\textbf{计算CPI:}
\begin{align*}
\text{Stall率} &= 0.30 \times 0.50 = 0.15 \\
\text{CPI} &= 1 + 0.15 \times 1 = 1.15
\end{align*}
\end{example}

\subsection{流水线性能公式}

\begin{formula}[吞吐量与加速比]
\textbf{吞吐量:}
\[
\text{Throughput} = \frac{n \text{ instructions}}{(k + n - 1) \text{ cycles}}
\]
其中 $k$ = 流水线级数, $n$ = 指令数

\textbf{加速比 (vs 非流水线):}
\[
\text{Speedup} = \frac{n \times k}{k + n - 1}
\]

当 $n \to \infty$: Speedup $\to k$
\end{formula}

\subsection{分支处理策略}

\begin{algorithm}[分支处理方法]
\textbf{1. Stall (最简单):}
\begin{itemize}
\item 等到分支结果确定再取下一条
\item 代价: 每个分支浪费若干周期
\end{itemize}

\textbf{2. 预测不跳转:}
\begin{itemize}
\item 总是取顺序下一条
\item 猜错时Flush流水线
\end{itemize}

\textbf{3. 延迟槽:}
\begin{itemize}
\item 分支后的指令总是执行
\item 编译器负责填充有用指令
\end{itemize}
\end{algorithm}

\begin{keybox}[考试重点]
\begin{itemize}
\item 会画流水线时序图
\item 能识别RAW/WAW/WAR冲突
\item 理解Forwarding路径
\item 知道Load-Use必须Stall
\item 计算CPI和加速比
\end{itemize}

\textbf{老师确认: 会考流水线,不考分支预测器细节}
\end{keybox}
