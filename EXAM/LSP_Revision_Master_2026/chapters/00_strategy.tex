% ==============================================================================
% 第零章: 考试策略与题型优先级表
% ==============================================================================
\section{考试策略速览}

\begin{keybox}[考试范围确认 (来自老师邮件)]
\textbf{必考内容:} 有符号/无符号数、K-Map化简、RS锁存器、香农展开、流水线\\
\textbf{不考内容:} \sout{Cache计算}、\sout{分支预测器}
\end{keybox}

\subsection{题型优先级矩阵}

\begin{center}
\small
\begin{tabular}{|l|c|c|c|}
\hline
\textbf{题型} & \textbf{分值} & \textbf{难度} & \textbf{优先级} \\
\hline
有符号/无符号转换 & 10-15 & 低 & [S+] 必拿 \\
\hline
K-Map 4变量 & 12-18 & 中 & [S] 必拿 \\
\hline
RS锁存器状态分析 & 8-12 & 中 & [A] 高分 \\
\hline
香农展开推导 & 10-15 & 中高 & [A] 高分 \\
\hline
流水线Hazard分析 & 15-20 & 中高 & [A] 高分 \\
\hline
\end{tabular}
\end{center}

\subsection{时间分配建议 (90分钟)}

\begin{algorithm}[时间分配策略]
\begin{enumerate}
\item \textbf{数值系统题} (15分钟): 快速完成,不要犯低级错误
\item \textbf{K-Map题} (20分钟): 画图仔细,检查圈是否正确
\item \textbf{RS锁存器/香农} (25分钟): 按步骤推导,写清楚过程
\item \textbf{流水线题} (25分钟): 画时序图,标注Hazard
\item \textbf{检查} (5分钟): 检查数值计算和符号
\end{enumerate}
\end{algorithm}

\begin{pitfall}[考场常见失分点]
\begin{itemize}
\item 有符号数忘记考虑符号扩展
\item K-Map圈的大小不是2的幂次
\item RS锁存器S=R=1的不稳定状态处理
\item 香农展开时代入0/1方向搞反
\item 流水线Forwarding路径画错
\end{itemize}
\end{pitfall}
